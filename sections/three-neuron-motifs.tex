\subsection{Three-neuron motif statistics}

There are 16 %
% %------------------------------------
% \footnote{%
%   There are 16 non-isomorphic simple directed graphs with 3
%   nodes. Three of those graphs are unconnected \parencite[cf. ][%
%   N. J. A. Sloane. The On-Line Encyclopedia of Integer Sequences,
%   http://oeis.org. Sequence
%   \href{http://oeis.org/A000273}{A000273}]{Davis1953}.%
% } %
% %-------------------------------------
non-isomorphic 3-motifs %?? by the strict definition 1-3 are not
                        %motifs
in simple directed graphs. In accordance with the study of Song et
al., the patterns were labeled 1 to 16. Assuming independence, the
expected distribution for the random variable $X$, that maps three
random vertices $v_1 \neq v_2 \neq v_3$ in a graph $G$ to the $n \in
\{1,2,\dots,16\}$, labeling the isomorphism class of the full subgraph
with vertex set $\{v_1,v_2,v_3\}$ in $G$ as above, can be obtained
from pair connection statistics. In anisotropic networks we found that
the probabilities of occurrence are
\begin{align*} 
  p_u & = 0.791336     &&\text{for unconnected pairs,}     \\
  p_s & = 0.184151     &&\text{for pairs with a single connection and} \\
  p_r & = 0.024513     &&\text{for reciprocally connected pairs.}
\end{align*}
We denote with $p_{\bar{s}} = p_s/2$ the probability to find a single
connection from $v_1$ in $v_2$ in a vertex pair $(v_1,v_2)$. The
probability of occurrence of the motif with label \enquote{8} is then
the product of the probabilities of it constituents multiplied with a
factor
\[
  \mathbf{P}(X=8) = 6\, p_{u} p_{s} p_{r},
\]
where the factor 6 is determined by the number of different
\textit{labeled} graphs belonging to the isomorphism class. The
distribution of $X$ for the remaining motifs is given by \\
%
\smallskip
%
\begin{minipage}{\linewidth}
  \begin{minipage}[c]{0.32\textwidth}
    \begin{align*}
      \mathbf{P}(X=1) &    =   p_u^3  \\
      \mathbf{P}(X=2) &    =   6 p_u p_u p_{\bar{s}}\\
      \mathbf{P}(X=3) &    =   3 p_u p_u p_r\\
      \mathbf{P}(X=4) &    =   3 p_{\bar{s}}^2 p_u\\
      \mathbf{P}(X=5) &    =   3 p_{\bar{s}}^2 p_u\\
    \end{align*}
  \end{minipage}%
  \begin{minipage}[c]{0.32\textwidth}
    \begin{align*}
      \mathbf{P}(X=\,\,\,6) &    =   6 p_{\bar{s}}^2 p_u\\
      \mathbf{P}(X=\,\,\,7) &    =   6 p_{\bar{s}} p_u p_r\\
      \mathbf{P}(X=\,\,\,9) &    =   3 p_r^2 p_u\\
      \mathbf{P}(X=10) &   =   6 p_{\bar{s}}^3   \\
      \mathbf{P}(X=11) &   =   2 p_{\bar{s}}^3    \\
    \end{align*}
  \end{minipage}%
  \begin{minipage}[c]{0.32\textwidth}
    \begin{align*}
      \mathbf{P}(X=12) &   =   3 p_{\bar{s}}^2 p_r\\
      \mathbf{P}(X=13) &   =   6 p_{\bar{s}}^2 p_r\\
      \mathbf{P}(X=14) &   =   3 p_{\bar{s}}^2 p_r\\
      \mathbf{P}(X=15) &   =   6 p_{\bar{s}} p_r^2\\
      \mathbf{P}(X=16) &   =   p_r^3.\\
    \end{align*}
  \end{minipage}  
\end{minipage}

\textbf{Verification of the distribution}

There $4^3 = 64$ possible combinations and summing up the coefficients
for motifs 1-16 yields exactly this number. 

It is
\[\sum_{i=1}^{16} \mathbf{P}(X=i) = (p_u+2p_{\bar{s}}+p_r)^3 = 1\]
as expected. % This is proven in research/mathematica/three_motif_distribution
