\subsection{Anisotropic network model -- formal definition}
\label{sec:aniso_model}

The anisotropic network model can be formally defined as a random
graph model, in which a realization of the random process results in a
geometric directed graph with a special mode of connectivity. For a
reference on directed graphs see for example \textcite{Bang-Jensen2002}.

\begin{definition*}
  Let $n \in \mathbb{N}$ and $w > 0$. An \textit{anisotropic geometric
    graph} $G_{n,w}$ consists of a tuple $(G,\Phi,a)$, of a simple
  directed graph $G$ with $|V(G)|=n$ vertices and the maps
  $\Phi:V(G)\to[0,1]^2$ and $a:V(G)\to[0,2\pi)$, such that for every
  pair of non-identical vertices $v,v' \in V(G)$ an edge $e\in E(G)$
  from $v$ to $v'$ exists if and only if the inequalities for scalar
  products
  \[
    R_{-a(v)}\left(\Phi(v')-\Phi(v)\right)\hat{e}_x \geq 0 
      \quad \mathrm{and} \quad
    \abs{R_{-a(v)}\left(\Phi(v')-\Phi(v)\right)\hat{e}_y} 
      \leq \frac{w}{2}
  \]
  hold. Here $R_{\varphi}$ is the rotation matrix of angle $\varphi$
  in the Cartesian plane,
  \[
   R_{\varphi} =  \begin{pmatrix}
      \cos \varphi & -\sin \varphi \\  
      \sin \varphi & \cos \varphi \\
    \end{pmatrix},
  \]
  and $\hat{e}_x = (1,0)$, $\hat{e}_y = (0,1)$ are the standard
  unit vectors.
\end{definition*}

The anisotropic random graph model is then giving the probability
distribution over the set of anisotropic random graphs by describing a
random process generating such graph.

\begin{definition*}
  Let $n \in \mathbb{N}$ and $w > 0$. The \textit{anisotropic random
    graph model} $G(n,w)$ is a probability space over the set of
  anisotropic geometric graphs with a probability distribution induced
  by the following process: Let $G$ be an empty graph with $n$
  vertices. Assign randomly and uniformly to every vertex $v \in V(G)$
  a position $\Phi(v) \in [0,1]^2$ and axonal orientation
  $0\leq a(v) < 2\pi$. Then add edges such that $(G,\Phi,a)$ is an
  anisotropic geometric graph $G_{n,w}$.
\end{definition*}


We here limited the surface to be the unit square. However, as only
the ratio between $w$ and the side length of the square $E$ determines
connectivity, this doesn't provide a real limitation as $w$ can always
be scaled appropriately to obtain equivalent graphs.
