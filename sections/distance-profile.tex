\subsection{Anisotropic network model}
\label{sec:aniso_model}

We will refer to the model as the \textit{anisotropic geometric
  network model}. Trying to provide a simple, abstract model isolating
anisotropy in connectivity, in most of this study the width $w(x)$ is
assumed to be constant, $w(x) =w$, a notable exception being the
development of tuned networks in Section~\ref{sec:tuned_networks}. In
the graph theoretic context the anisotropic network model is a random
graph model, in which a realization of the random process results in a
geometric directed graph with a special mode of connectivity. We can
formally define such realization as:

\begin{definition}[Anisotropic geometric graph]
  \label{def:anisotropic_network} 
  Let $n \in \mathbb{N}$ and $w \in (0,\infty)$. An
  \textit{anisotropic geometric graph} $G_{n,w}$ then consists of a
  tuple $(G,\Phi,a)$, of a simple directed graph $G$ with $|V(G)|=n$
  vertices and the maps $\Phi:V(G)\to[0,1]^2$ and $a:V(G)\to[0,2\pi)$,
  such that for every vertex pair $v,v' \in V(G)$ and edge $e\in E(G)$
  with $s(e)=v$ and $t(e)=v'$ exists if and only if the inequalities
  for scalar products
  \[
    R_{-a(v)}\left(\Phi(v')-\Phi(v)\right)\hat{e}_x \geq 0 
      \quad \mathrm{and} \quad
    \abs{R_{-a(v)}\left(\Phi(v')-\Phi(v)\right)\hat{e}_y} 
      \leq \frac{w}{2}
  \]
  hold. Here $R_{\varphi}$ is the rotation matrix of angle $\varphi$
  in the Cartesian plane and $\hat{e}_x, \hat{e}_y$ are the standard
  unit vectors. % ?? Alpha or A??
  % for $y=A_{\alpha(v)}(v'-v)$ the identities $y\hat{e}_x > 0$
  % and $\abs{y \hat{e}_y} \le \frac{w}{2}$ hold.
\end{definition}

The anisotropic random graph model then is then giving the probability
distribution over the set of anisotropic random graphs by describing a
random process generating such graph.

\begin{definition}[Anisotropic random graph model]
  \index{anisotropic geometric!random graph model} 
  Let $n \in \mathbb{N}$ and $w > 0$. The \textit{anisotropic random
    graph model} $G(n,w)$ is a probability space over the set of
  anisotropic geometric graphs with a probability distribution induced
  by the following process: Let $G$ be an empty graph with $n$
  vertices. Assign randomly and uniformly to every vertex $v \in V(G)$
  a position $\Phi(v) \in [0,1]^2$ and axonal orientation $0\leq a(v)
  < 2\pi$. Then add edges such that $(G,\Phi,a)$ is an anisotropic
  geometric graph $G_{n,w}$.
\end{definition}


As with every geometric graph model introduced, we restrict the
surface to be the unit square. This does not limit the model, as only
the relative width of the axon band in regard to the surface's side
length is determining connectivity statistics -- the expected number
of connections is easily obtained by the quotient of the area covered
by the axon and the surface area, making connectivity statistics in
the anisotropic random graph model essentially \enquote{scale-free}.

The following maybe interpreted as a study of anisotropic geometric
graphs in the light of a neuroscientific context. To enable such an
analysis, a few more concepts are needed. The introduction of
those concepts composes the rest of the chapter. A first important
step is the numerical implementation of the anisotropic network model.


\subsection{Anisotropic network -- Distance-dependent connection probability}

\begin{theorem*} \label{theorem:distance_prof} Let $G_{n,w} = (G,\Phi,
  a)$ be an anisotropic random graph. Define $C:[0,\sqrt{2}] \to
  [0,1]$ as the distance-dependent connection probability profile of
  $(G,\Phi)$, that is such that $C(x)$ is the probability that for a
  vertex pair $(v,v') \in V(G)^2\setminus\Delta_{V(G)}$ in distance $x
  = \norm{\Phi(v)-\Phi(v')}$ the vertex $v$ projects to vertex
  $v'$. Then
  \[
    C(x) = \begin{cases}%
             \frac{1}{2} & \mathrm{for} \,\, x\le w/2 \\
             \frac{1}{\pi}
             \operatorname{arcsin}\left(\frac{w}{2x}\right) &
             \mathrm{for} \,\, x >  w/2. %
           \end{cases}
  \]
\end{theorem*} 

\begin{proof}
  Let $v,v'$ be a pair of vertices in $V(G)^2 \setminus \Delta_{V(G)}$
  in Euclidean distance $x$ of each other. The vector difference
  $\Phi(v')-\Phi(v)$ may then be written as $x e^{i\theta}$, with $0
  \leq \theta < 2\pi$. We have 
  \[
    R_{-\alpha(v)} xe^{i\theta} = xe^{i(\theta - \alpha(v))}.
  \]
  Only for suitable combination of $\theta$ and $\alpha(v)$ an edge
  from $v$ to $v'$ exists. Assuming $\alpha(v)$ fixed, we calculate
  the probability of connection depending on the random choice of
  $\theta$. We can assume $\alpha(v) = 0$, otherwise the same argument
  holds for $\theta' = \theta - \alpha(v)$.

  From the definition of the anisotropic random graph we obtain the
  necessary and sufficient conditions
  \[
   x \cos \theta \geq 0 \quad \mathrm{and} \quad \abs{x\sin \theta}
  \leq \frac{w}{2}.
  \]
  Choosing uniformly at random $\theta$ in the range of $[0,2\pi)$,
  the first condition is satisfied with a probability of
  $\frac{1}{2}$. Consider for the second condition $\theta \in
  [0,\pi)$. We have 
  \[ 
  \sin \theta \leq \frac{w}{2x},
  \]
  and for $x \leq \frac{w}{2}$ the inequality holds for all $\theta$
  by definition of $\sin \theta$. In the case of $x > \frac{w}{2}$, we
  note that for the first condition to hold $\theta$ must already be in
  $[0,\frac{\pi}{2})$ and can thus write the second condition $\theta$ as
  \[
    \theta \leq \operatorname{arcsin}\frac{w}{2x},
  \]
  yielding $C(x)$ by combining the considerations above and using the
  symmetry of sine for $\theta$ in the third and fourth quadrant.
  % 
\end{proof}


\vspace{0.2cm}%??
\addtocounter{subsection}{1}
\begin{figure}[h!] 
  \centering 
  \includegraphics[width=0.85\textwidth]{%
    SI_geomtr-prb/SI_geomtr-prb.pdf}%
  \caption{\textbf{Illustrating the proof of
      Theorem~\ref{theorem:distance_prof}} Distance-dependent
    connectivity profile $C(x)$ in an anisotropic geometric graph
    calculated from geometric dependencies. \textbf{A)} In the case of
    $x\leq \nicefrac{w}{2}$, target $v'$ may be located anywhere on the
    shown semicircle and therefore receives input from $v$ with
    probability $\nicefrac{1}{2}$. \textbf{B)} For $x > \nicefrac{w}{2}$,
    suitable positions for target $v'$ are dependent on $x$. The
    geometric relation $\sin \theta = \nicefrac{w}{2x}$ leads to the
    distance-dependent connectivity profile as described in
    Theorem~\ref{theorem:distance_prof}.}
  \label{fig:xx_geomtr_prb}
\end{figure}

