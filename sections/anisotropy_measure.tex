\subsection{Anisotropy degree}
\label{sec:aniso_measure}

The anisotropy degree $\lambda$ can be defined formally as follows:

\begin{definition*}
  Let $(G, \Phi, a)$ be an anisotropic geometric graph. For each
  vertex $v$ in $G$, with non-empty set of targets $T(v)$ of $v$, the
  anisotropy degree of $v$ is defined as
  \[
    \lambda(v) =       \frac{1}{\abs{T(v)}} \norm{\sum_{w \in T(v)}
    \frac{\Phi(w)-\Phi(v)}{\norm{\Phi(w)-\Phi(v)}}},
  \]
  where $\abs{T(v)}$ is the number of targets of $v$ and
  $\norm{\cdot}$ denotes the euclidean norm.
\end{definition*}

By the triangle inequality, the anisotropy degree $\lambda(v)$ takes
values from $0$ to $1$. If targets of $v$ mostly align along a
projection from $v$, the degree $\lambda(v)$ is close to $1$. On
contrast, if targets are widely dispersed, $\lambda(v)$ is almost
$0$. Note however that $\lambda(v) = 0$ does not necessarily imply
even distribution of directions, examples for this are easily
constructed \cite[cf.][]{Mardia2000}.


