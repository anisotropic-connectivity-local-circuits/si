\subsection{Tuning Distance-dependency}\label{sec:tuned_networks}

The discussion in the last section focused on the effect of anisotropy
in connectivity on the occurrence of neuron pair motifs. Could
distance-dependency itself, as imposed by the specific geometry, be a
decisive factor in the distribution of edge counts in neuron pairs?
%% \textcite{Song2005}, as well as \textcite{Perin2011}, report an
%% overrepresentation of reciprocal connections independent from
%% distance-dependent connectivity, opposing the observations made in the
%% last section (\autoref{fig:two_neuron_probs} A). Furthermore, the
%% connectivity profile in the anisotropic graph model, as identified in
%% Section~\ref{sec:distance_connectivity}, follows purely from abstract
%% geometry rather than being motivated by connectivity found in cortical
%% circuits. In an attempt to rectify this and to allow for a more
%% differentiated examination of two neuron connections, in this section
%% we step away from simplistic geometry and \enquote{tune} the
%% anisotropic networks to display a distance-dependent connectivity as
%% reported by Perin et al. by adjusting the width $w(x)$ at any point
%% $x$ along the axon's projection.

%% For this we introduce anisotropic networks tuned to reflect a given
%% distance-dependent connection profile $C(x)$. We are facing the
%% following problem: Given $C(x):[0,\sqrt{2}) \to [0,1]$, find
%% $w:[0,\sqrt{2}) \to [0,\infty)$ such that the probability to have a
%% connection from $v_1$ to $v_2$ for arbitrary vertices $v_1 \neq v_2$
%% in an anisotropic graph $G(n,w)$ with distance $\mathrm{d}(v_1,v_2) =
%% x$ is $C(x)$. The problem is in general highly complex when nothing
%% can be assumed about $C(x)$. We find an approximate solution to the
%% problem considering the following geometric relation:

%% \begin{figure}[htp]
%%   \centering
%%   \makebox{%
%%     \begin{overpic}[height=3.35cm]{%
%%         plots/bed7650b.pdf}
%%     \end{overpic}
%%   }%
%%   \caption{Computing connection probability $C(x)$ from non-constant
%%     $w(x)$}
%%   \label{fig:dpp_wc}
%% \end{figure}

%% From \autoref{fig:dpp_wc} we have the relation  
%% \begin{equation}
%% C\left(\sqrt{x^2+w^2(x)}\right) = \frac{1}{\pi} \operatorname{arctan}
%% \frac{w(x)}{x}. \label{eq:geo_rel}
%% \end{equation} 
%% In order to solve for $w(x)$ we first consider a linear approximation,
%% expanding
%% \[C\left(\sqrt{x^2+w^2(x)}\right) \approx C(x) + \left(\sqrt{x^2+w^2(x)} -
%% x\right) C'(x).\]

%% The resulting transcendental equation
%% \[C(x) + \left(\sqrt{x^2+w^2(x)} -
%% x\right) C'(x) = \frac{1}{\pi} \operatorname{arctan}
%% \frac{w(x)}{x}\]
%% is however still too complex in the context of this work. Instead we
%% propose the approximation $\sqrt{x^2 + w^2(x)} \approx  x$, which
%% inserting into \ref{eq:geo_rel} yiels
%% \begin{equation}
%% C(x) \approx \frac{1}{\pi} \operatorname{arctan} \label{eq:tanapprox}
%% \frac{w(x)}{x}.
%% \end{equation}

%% Under the assumption that $C(x)<\frac{1}{2}$ for all $x$ we obtain the
%% identity
%% \begin{equation}
%%   w(x) = x \tan\left( \pi\, C(x) \right), \label{eq:xtan}
%% \end{equation} 
%% being aware that it only holds as well as
%% approximation~\ref{eq:tanapprox} does. 

%% Here we use relation~\ref{eq:xtan} to generate anisotropic networks
%% reflecting the dis\-tance-de\-pendent connectivity profile as found by
%% \textcite{Perin2011}. For this we finally need to adjust the before
%% arbitrarily determined side length of the network's surface. Perin et
%% al.~mapped connectivity in layer 5 of the rat's somatosensory cortex
%% up to a distance of $\SI{300}{\micro\meter}$. Using this reported
%% distance connectivity to generate anisotropic networks via
%% \ref{eq:xtan}, the chosen side length $s$ determines the networks
%% overall connectivity (\autoref{fig:determine_side_length} A). We
%% determine $s = \SI{296}{\micro\meter}$ to match the overall connection
%% probability of $p = 0.116$ as used before and reported by Song et
%% al.~(\autoref{fig:determine_side_length} B). The obtained value for
%% $s$ is consistent with the slice thickness of \SI{300}{\micro\meter}
%% used in Perin et al.'s experiment.


%% %% \begin{figure}[htp]
%% %%   \centering
%% %%   \makebox{%
%% %%     \begin{overpic}[height=4.05cm]{%
%% %%         plots/6154302f.pdf}
%% %%       \put(85.5,57.0){\small\textbf{A}}
%% %%       %\put(12,5){\small\textbf{A}}
%% %%     \end{overpic}
%% %%     \hfill
%% %%     \begin{overpic}[height=3.955cm]{%
%% %%         plots/ef0e785d.pdf}
%% %%       \put(88.5,58.2){\small\textbf{B}}
%% %%     \end{overpic}
%% %%   }%
%% %%   \captionsetup{skip=7pt}
%% %%   \caption{\textbf{Network side length adjusted to match overall
%% %%       connection probability} Side length of the network's surface
%% %%     determines the overall connection probability in the network when
%% %%     axon width function $w(x)$ is fixed. \textbf{A)} Connection
%% %%     probability declines with rising side length \textbf{B)}
%% %%     Determining side length as $s=\SI{296}{\micro\meter}$ to match $p
%% %%     = 0.116$ as reported by \textcite{Song2005}. (\smtcite{6154302f},
%% %%     \smtcite{ef0e785d})}
%% %%   \label{fig:determine_side_length}
%% %% \end{figure}



%% Having determined the neotwork's side length $s$, we're extending the
%% quiver of generated sample networks for the numerical analysis once
%% more by the \enquote{tuned anisotropic graphs}\index{tuned
%%   anisotropic networks}, in which the axon width $w(x)$ was determined
%% such that the networks reflect Perin's connectivity profile. Analyzing
%% the obtained axon width function we note that $x \gg w(x)$ holds for
%% most $x$, justifying the approximation
%% \[
%%   \sqrt{x^2 + w^2(x)} \approx x
%% \] 
%% \textit{a posteriori} (\autoref{fig:perin_axwidth}). From the 25
%% generated networks overall connection probability is extracted as $p =
%% 0.1160 \pm 0.0006$ (SEM), as expected from the choice of $s$
%% (\smtcite{f11dca65}).



%% % This approximation holds well as long as $x \gg w(x)$. Using the
%% % relation to tune the axon width to produce anisotropic networks with a
%% % distance-dependency as reported by Perin et al., we find that for all $x$ is
%% % strictly greater than $w(x)$  %(\autoref{fig:perin_axwidh


%% \begin{figure}[htp]
%%   \centering
%%   \hspace{0.05cm}
%%   \begin{overpic}[width=0.6\textwidth]{%
%%       plots/d45c02e4.pdf}
%%           \put(69.4,51.5){\small\textbf{A}}
%%   \end{overpic}
%%   \hfill
%%   \begin{overpic}[width=0.35\textwidth]{%
%%       plots/8f0d65e4.pdf}
%%     \put(81,86){%
%%       \fboxsep=2pt\colorbox{white}{\small\textbf{B}}
%%     }
%%   \end{overpic}
%%   \captionsetup{skip=7pt}
%%   \caption{\textbf{Anisotropic network model with tuned axon width
%%       $\mathbf{w(x)}$} \textbf{A)} Resulting axon width function
%%     $w(x)$ from tuning to distance-dependent connection profile as
%%     reported by \textcite{Perin2011}, see also
%%     \autoref{fig:perin_profiles}. Note that $x \gg w(x)$ for most $x$,
%%     supporting approximation~\ref{eq:tanapprox}. \textbf{B)}
%%     Showing for a single neuron (star) connected (red) and unconnected
%%     (gray) neurons in the tuned anisotropic network, revealing
%%     the characteristic axon shape. (\smtcite{d45c02e4}, \smtcite{8f0d65e4})}
%%   \label{fig:perin_axwidth}
%% \end{figure}




%% Overall distance-dependent connection probabilities in the tuned
%% an\-iso\-tro\-pic graphs clearly match the profile of Perin et
%% al.~(\autoref{fig:perin_profiles} A), presenting strongest the
%% argument in support of the chosen approximation. Analyzing two neuron
%% connections \marginpar{revisiting two neuron connections} in the tuned
%% networks, we affirm the findings of the last section. In their
%% experiment, Perin et al.~were able to show an overrepresentation of
%% reciprocal connections at any inter-neuron distance
%% (\autoref{fig:perin_profiles} B-C). Rather than matching these
%% profiles, we find that occurrences of one- and bidirectionally
%% connected pairs in the anisotropic graphs align with probabilities
%% obtained from the distance-dependent overall connection probability
%% $p(x)$ under the assumption of independence (cf. Equation~\ref{eq:pairs}),
%% \begin{equation*}
%%   \label{eq:pairs}
%%   \begin{aligned}%
%%     & \mathbf{P}_{X=1}(x) = 2p(x) \left(1-p(x) \right)    
%%       && \text{single connection,}\\
%%     & \mathbf{P}_{X=2}(x) = p(x)^2        
%%       &&\text{reciprocal connection.}
%%   \end{aligned}%
%% \end{equation*}%
%% \vspace{0.1cm}%
%% Thus, in comparison with Perin et al.'s findings, we find that
%% anisotropy in connectivity cannot account for the overrepresentation
%% in reciprocal connections. While results in
%% Section~\ref{sec:two_neuron} still indicated such an
%% overrepresentation due to distance-dependency, examining the
%% occurrence of two neuron connections at any inter-neuron distance in
%% anisotropic networks, tuned to a distance-dependent connection profile
%% matching experimental findings from cortical circuits, imply complete
%% unrelatedness of anisotropy and two-neuron connection distributions.

%% \begin{figure}[htp]
%%   \centering
%%   \makebox{%
%%     \begin{overpic}[width=0.5\textwidth]{%
%%         plots/875505b0_overall.pdf}
%%       \put(28,19){\small\textbf{A}}
%%     \end{overpic}
%%     \hfill
%%     \begin{overpic}[width=0.5\textwidth]{%
%%         plots/875505b0_single.pdf}
%%       \put(28,19){\small\textbf{B}}
%%     \end{overpic}
%%   }%
%%   \vspace{-0.6cm}
%%   \makebox{%
%%     \begin{overpic}[width=0.5\textwidth]{%
%%         plots/875505b0_recip.pdf}
%%        \put(28,19){\small\textbf{C}}
%%     \end{overpic}
%%     \vspace{-1cm}
%%     \includegraphics[width=0.5\textwidth]{%
%%       img/tuned_legend.pdf}   
%%   }%
%%   \captionsetup{skip=7pt}
%%   \caption{\textbf{Distance-independent overrepresentation of
%%       reciprocal connections} Comparison of occurrences of one- and
%%     bidirectionally connected neuron pairs in the tuned anisotropic
%%     networks (gray) with profiles found by Perin et al.~(red), shows
%%     that overrepresentation of bidirectional pairs is
%%     distance-independent and not connected to anisotropy.  \textbf{A)}
%%     Overall connection probability in the tuned anisotropic networks
%%     was successfully adjusted to reflect connection probability found
%%     by Perin et al. \textbf{B)-C)} Showing in blue the probabilities
%%     to obtain a neuron pair motif (single edge in B, two edges in C)
%%     calculated under independence assumption from the overall
%%     probability from A), we find that counts in the tuned anisotropic
%%     networks (gray) match the independence assumption and do
%%     \textit{not} show the overrepresentation present in Perin et al.'s
%%     experiment. (\smtcite{875505b0})}
%%   \label{fig:perin_profiles}
%% \end{figure}



