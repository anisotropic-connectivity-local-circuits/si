
\bigskip

\subsection{Anisotropic network -- Distance-dependent connection probability}

The formal definition of anisotropic graphs in \ref{sec:aniso_model}
allows us for calculation of the distance-dependent connection
probabilities in the network model.

\begin{theorem_w} \label{theorem:distance_prof} Let $G_{n,w} = (G,\Phi,
  a)$ be an anisotropic random graph. Define $C:[0,\sqrt{2}] \to
  [0,1]$ as the distance-dependent connection probability profile of
  $(G,\Phi)$, that is such that $C(x)$ is the probability that for a
  vertex pair $(v,v') \in V(G)^2\setminus\Delta_{V(G)}$ in distance $x
  = \norm{\Phi(v)-\Phi(v')}$ the vertex $v$ projects to vertex
  $v'$. Then
  \[
    C(x) = \begin{cases}%
             \frac{1}{2} & \mathrm{for} \,\, x\le w/2 \\
             \frac{1}{\pi}
             \operatorname{arcsin}\left(\frac{w}{2x}\right) &
             \mathrm{for} \,\, x >  w/2. %
           \end{cases}
  \]
\end{theorem_w} 

\begin{proof}
  Let $v,v'$ be a pair of vertices in $V(G)^2 \setminus \Delta_{V(G)}$
  in Euclidean distance $x$ of each other. The vector difference
  $\Phi(v')-\Phi(v)$ may then be written as $x e^{i\theta}$, with $0
  \leq \theta < 2\pi$. We have 
  \[
    R_{-\alpha(v)} xe^{i\theta} = xe^{i(\theta - \alpha(v))}.
  \]
  Only for suitable combination of $\theta$ and $\alpha(v)$ an edge
  from $v$ to $v'$ exists. Assuming $\alpha(v)$ fixed, we calculate
  the probability of connection depending on the random choice of
  $\theta$. We can assume $\alpha(v) = 0$, otherwise the same argument
  holds for $\theta' = \theta - \alpha(v)$.

  From the definition of the anisotropic random graph we obtain the
  necessary and sufficient conditions
  \[
   x \cos \theta \geq 0 \quad \mathrm{and} \quad \abs{x\sin \theta}
  \leq \frac{w}{2}.
  \]
  Choosing uniformly at random $\theta$ in the range of $[0,2\pi)$,
  the first condition is satisfied with a probability of
  $\frac{1}{2}$. Consider for the second condition $\theta \in
  [0,\pi)$. We have 
  \[ 
  \sin \theta \leq \frac{w}{2x},
  \]
  and for $x \leq \frac{w}{2}$ the inequality holds for all $\theta$
  by definition of $\sin \theta$. In the case of $x > \frac{w}{2}$, we
  note that for the first condition to hold $\theta$ must already be in
  $[0,\frac{\pi}{2})$ and can thus write the second condition $\theta$ as
  \[
    \theta \leq \operatorname{arcsin}\frac{w}{2x},
  \]
  yielding $C(x)$ by combining the considerations above and using the
  symmetry of sine for $\theta$ in the third and fourth quadrant.
  % 
\end{proof}


\vspace{1.6cm}
\addtocounter{subsection}{1}
\begin{figure}[h!] 
  \centering 
  \includegraphics[width=0.65\textwidth]{%
    SI_geomtr-prb/SI_geomtr-prb.pdf}%
  \caption{\textbf{Illustration of the proof} Distance-dependent
    connectivity profile $C(x)$ in an anisotropic geometric graph
    calculated from geometric dependencies. \textbf{A}: In the case of
    $x\leq \nicefrac{w}{2}$, target $v'$ may be located anywhere on
    the shown semicircle and therefore receives input from $v$ with
    probability $\nicefrac{1}{2}$. \textbf{B}: For
    $x > \nicefrac{w}{2}$, suitable positions for target $v'$ are
    dependent on $x$. The geometric relation
    $\sin \theta = \nicefrac{w}{2x}$ leads to the distance-dependent
    connectivity profile as described above.}
  \label{fig:xx_geomtr_prb}
\end{figure}

