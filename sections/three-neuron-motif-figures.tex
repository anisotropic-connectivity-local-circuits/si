
\addtocounter{subsection}{1}
\begin{figure}[h!]
  \includegraphics[width=\textwidth]{%
    /home/fh/sci/rsc/aniso_netw/pub/arxiv18/figures/SI_3motifs/rel-rew-dist_1x1.pdf}
  \caption{Over- and underrepresentation induced in anisotropic
    networks relative to their rewired versions. In both anisotropic
    and tuned anisotropic networks specific motifs occur significantly
    more often than in their rewired versions. For 5 anisotropic
    network instances (blue) the occurrence of three motifs was
    recorded in $n=300000$ random groups of three neurons. The
    networks were then rewired and triplet counts were recorded in
    $n=300000$ random groups of three neurons. Shown is the difference
    between the counts in anisotropic networks and the rewired
    networks was divided by expected number of counts from
    probabilities of pairs in anisotropic networks. Data in red in red
    is for tuned anisotropic networks with the same procedure. Motifs
    labelled with * were reported by \textcite{Perin2011} to be
    overrepresented in layer 5 pyramidal cell of rat somatosensory
    cortex.}
  \label{fig:3motif_rel} 
\end{figure}


\addtocounter{subsection}{1}
\begin{figure}[h!]
  \includegraphics[width=\textwidth]{%
   /home/fh/sci/lab/aniso_netw/ploscb_18/fig/si/figSI_3motif_rew-rew5-rew10.pdf}
 \caption{Repeated rewiring does not change the statistics of three
   neuron motif occurrence. As in Fig.~\ref{MAIN-fig:3_motifs} the
   ratio between counts recorded and counts predicted from pair
   connectivity statistics is shown. In green for rewired anisotropic
   networks, in grey for anisotropic networks which had the rewiring
   algorithm applied 5 times, in magenta anisotropic networks that had
   the rewiring algorithm applied 10 times.}
  \label{fig:3motif_rr5r10} 
\end{figure}

